% Mengubah keterangan `Abstract` ke bahasa indonesia.
% Hapus bagian ini untuk mengembalikan ke format awal.
\renewcommand\abstractname{Abstrak}

\begin{abstract}

  % Ubah paragraf berikut sesuai dengan abstrak dari penelitian.
  Integritas dari sebuah File Digital merupakan salah satu aspek penting dari sekuriti sistem komputer. Terdapat instilah yang dinamakan dengan File Integrity Monitoring (FIM) yang merupakan sebuah proses yang melakukan tindakan validasi dari sebuah File pada Operating System dan software aplikasi menggunakan metode verifikasi antara state atau keadaan yang terkini dengan state yang diketahui atau sebelumnya. Tentu saja hal tersebut penting untuk dilakukan sehingga file-file yang selalu kita transmisikan baik melalui lokal komputer dan melalui nirkabel akan selalu terjamin isinya sehingga tidak korup. Apabila tidak demikian maka mungkin dapat terjadi kerusakan atau pemalsuan data. Maka pada makalah ini saya melakukan riset terhadap metode yang dapat menjamin integritas pada file digital di sistem komputer. Dari segi security, terdapat beberapa penelitian yang juga melibatkan perkara \emph{file integrity}. Penelitian dengan tema secure messaging juga memberikan kita penemuan yang relevan tentang usability dan security dari proses autentikasi pengguna layanan messaging tersebut. Parity Bit atau biasa juga disebut dengan \emph{Check Bit} adalah bentuk sederhana dari \emph{error detecting code}. Terdapat dua varian dari bit paritas, paritas genap dan paritas ganjil. Sebuah checksum merupakan suatu blok data berukuran kecil yang diperoleh dari blok data digital yang lainnya. Sebuah \emph{MD5 message-digest algorithm} adalah hash function yang sering digunakan untuk mengecek integritas pada file. MD5 digest sudah digunakan secara luas pada dunia perangkat lunak untuk memberikan sebuah jaminan dimana file yang ditransmisikan telah tiba dan datanya sama dengan data yang asli. SHA-1 merupakan fungsi \emph{hash cryptographic} yang menerima input lalu akan mengeluarkan output 160-bit (20 byte) nilai hash yang dikenal sebagai \emph{message digest}. SHA-2 merupakan perubahan yang cukup signifikan dibandingkan pendahulunya, SHA-1. SHA-3 adalah subset dari Keccak yang didasari dari pendekatan baru yang dinamakan sponge construction. Sponge construction didasari dari fungsi random atau fungsi permutasi. Sebuah \emph{Cyclic Redundancy Check} adalah \emph{error-detecting code} yang biasa digunakan pada jaringan digital dan storage untuk mendeteksi adanya perubahan yang tidak diinginkan pada data. Komputasi dari CRC diturunkan dari polynomial division, modulo dua.

\end{abstract}

% Mengubah keterangan `Index terms` ke bahasa indonesia.
% Hapus bagian ini untuk mengembalikan ke format awal.
\renewcommand\IEEEkeywordsname{Kata kunci}

\begin{IEEEkeywords}

  % Ubah kata-kata berikut sesuai dengan kata kunci dari penelitian.
  \emph{Integrity}, \emph{Security}, Komputer, File, \emph{MD5}, \emph{Checksum}, \emph{SHA}, \emph{Digest}, \emph{CRC}.

\end{IEEEkeywords}
