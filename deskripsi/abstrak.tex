% Mengubah keterangan `Abstract` ke bahasa indonesia.
% Hapus bagian ini untuk mengembalikan ke format awal.
\renewcommand\abstractname{Abstrak}

\begin{abstract}

  % Ubah paragraf berikut sesuai dengan abstrak dari penelitian.
  Integritas dari sebuah File Digital merupakan salah satu aspek penting dari sekuriti sistem komputer. Terdapat instilah yang dinamakan dengan File Integrity Monitoring (FIM) yang merupakan sebuah proses yang melakukan tindakan validasi dari sebuah File pada Operating System dan software aplikasi menggunakan metode verifikasi antara state atau keadaan yang terkini dengan state yang diketahui atau sebelumnya. Tentu saja hal tersebut penting untuk dilakukan sehingga file-file yang selalu kita transmisikan baik melalui lokal komputer dan melalui nirkabel akan selalu terjamin isinya sehingga tidak korup. Apabila tidak demikian maka mungkin dapat terjadi kerusakan atau pemalsuan data. Maka pada makalah ini saya melakukan riset terhadap metode yang dapat menjamin integritas pada file digital di sistem komputer.

\end{abstract}

% Mengubah keterangan `Index terms` ke bahasa indonesia.
% Hapus bagian ini untuk mengembalikan ke format awal.
\renewcommand\IEEEkeywordsname{Kata kunci}

\begin{IEEEkeywords}

  % Ubah kata-kata berikut sesuai dengan kata kunci dari penelitian.
  \emph{Integrity}, \emph{Security}, Komputer, File.

\end{IEEEkeywords}
