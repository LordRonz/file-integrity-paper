% Ubah judul dan label berikut sesuai dengan yang diinginkan.
\section{Kesimpulan}
\label{sec:kesimpulan}

% Ubah paragraf-paragraf pada bagian ini sesuai dengan yang diinginkan.

Dari semua bagian yang sudah diberikan pembahasannya pada makalah ini, dapat diberi kesimpulan sebagai berikut. Parity Bit adalah salah satu cara untuk melakukan uji integritas file yang paling sederhana, tetapi pada penggunaannya terdapat kelemahan-kelemahan seperti jumlah bit yang sama-sama genap maupun ganjil tetapi berbeda dari yang aslinya akan dianggap tidak benar. Parity bit sendiri memiliki kelebihan seperti tidak memerlukan beban komputasi yang besar. Checksum sendiri merupakan istilah yang lebih luas dari algoritma-algoritma yang digunakan untuk melakukan pengecekan integritas. Terdapat banyak algoritma untuk melakukan checksum, salah satunya yang populer adalah MD5. MD5 merupakan algoritma hash function yang umum digunakan, meski sekarang hanya digunakan untuk sekedar melakukan cek integritas pada file yang tidak dimodifikasi secara sengaja. Hal ini terjadi karena komputer sekarang makin cepat sehingga metode brute force untuk menemukan collision dapat dengan mudah dilakukan. MD5 adalah varian dari message digest algorithm. SHA-1 merupakan generasi kedua dari keluarga SHA (Secure Hash Algorithm). SHA-1 juga sudah ditinggalkan karena alasan keamanan yang sama dengan MD5. SHA-2 merupakan hash function algorithm yang paling umum digunakan sekarang. NIST masih merekomendasikan SHA-2, namun disarankan menggunakan hash value yang besar agar semakin aman. SHA-2 juga dapat digunakan untuk melakukan pengecekan integritas file, namun apabila tidak memedulikan aspek sekuriti tidak disarankan karena beban komputasi yang lebih tinggi. SHA-3 merupakan terobosan terbaru dari keluarga SHA, karena menggunakan pendekatan yang berbeda dari varian sebelumnya, Keccak Algorithm. SHA-3 diyakini lebih aman dalam masalah anti collision, namun belum secara luas digunakan. CRC atau Cyclic Redundancy Check merupakan salah satu jenis checksum namun tidak diperuntukkan untuk keamanan. CRC bagus digunakan untuk melakukan verifikasi saat menyalin file ataupun pengarsipan file. Algoritma yang digunakan CRC juga terbilang tidak serumit MD5 dan SHA, sehingga pada \emph{Integrated Ciruit} biasanya sudah diaplikasikan dan siap digunakan. 
